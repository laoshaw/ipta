% !TeX encoding = UTF-8
\documentclass[english,twoside,openright,a4paper,12pt]{article}
\usepackage{a4}
\usepackage[ansinew]{inputenc}
\usepackage[english]{babel}
\usepackage{graphics}
\usepackage{dcolumn}
\usepackage{fancyhdr}
\usepackage{amsmath}	
\usepackage{graphicx}
\usepackage{color}
\usepackage[hidelinks]{hyperref}
\usepackage{float} 
\restylefloat{table}
\restylefloat{figure}
\usepackage{color}
\usepackage{textcomp}
\usepackage{emptypage}
\usepackage[table]{xcolor}
\usepackage{index}
%\makeindex
%\makeglossary
\newcommand{\hilight}[1]{\colorbox{yellow}{#1}}
\nonfrenchspacing
\renewcommand{\baselinestretch}{1.0}
\clubpenalty=9999    % Not higher!
\widowpenalty=9999   % Not higher!
\setcounter{secnumdepth}{4}
\setcounter{tocdepth}{3}
\newcommand{\titlerule}{\rule{\linewidth}{1pt}}
\renewcommand{\headrulewidth}{1.2pt}
\renewcommand{\footrulewidth}{1.2pt}
\newcommand{\ai}[1]{#1\index{#1}}

\author{Anders Sikvall}
\title{
	\Huge ipta\\[2em]
	IP Tables Log Analyzer}

\begin{document}

\begin{titlepage}
	\thispagestyle{empty}
	\begin{center}
          % Logotype goes here
	\end{center}
	\vspace*{\stretch{1}}
	\begin{center}
		\titlerule\\[3mm]
		\Huge \textsc{ipta\\\Large iptables Log Analyzer}\\[5mm]
		\large \emph{Anders Sikvall}\\
		\begin{center}
			\normalsize ichimusai.org\\

		\end{center}
		\titlerule\\
	\end{center}
	%    \scriptsize \today
	\vspace*{\stretch{2}}
	%    \begin{center}
	%         \normalsize\textcopyright Copyright DeltaNode 2015
	%    \end{center}
\end{titlepage}

\null\vfill\thispagestyle{empty}
\noindent

\begin{center}
	Version 0.1\\[5mm] \textcopyright\ Copyright 2015 Anders Sikvall, 
	ichimusai.org\\
	Printed in Sweden
	\newpage
\end{center}

\cleardoublepage

\tableofcontents

% Turn on more technical formatting, this aint some kind of novella
% you know
\setlength{\parindent}{0pt}
\setlength{\parskip}{1em}

\lhead{\nouppercase{\leftmark}}
\rhead{\nouppercase{\rightmark}}

\pagestyle{fancy}

%%%%%%%%%%%%%%%%%%%%%%%%%%%%%%%%%%%%%%%%%%%%%%%%%%%%%%%%%%%%%%%%%%%%%%

\newpage
\section*{License}
\label{license}

Copyright (c) 2014, Anders "Ichimusai" Sikvall\\
Latest revision 2014-10-05
 
Permission is hereby granted, free of charge, to any person obtaining
a copy of this software and associated documentation files (the
"Software"), to deal in the Software without restriction, including
without limitation the rights to use, copy, modify, merge, publish,
distribute, sublicense, and/or sell copies of the Software, and to
permit persons to whom the Software is furnished to do so, subject to
the following conditions:

\begin{itemize}
\item[1.] The above copyright notice and this permission notice shall
  be included in all copies or substantial portions of the Software.
  It is not allowed to modify this license in any way.
  
\item[2a.] The original header must accompany all software files and
  are not removed in redistribution. It is allowed to add to the
  headers to describe modifications of the software and who has made
  these modifications. I claim no right to your modifications, they
  stand on their own merits.
 
\item[2b.] The license shown when the software is invoked with the
  "--license" option is not removed or modified but must remain the
  same. You may add your name to the bottom of the credit part if you
  have contributed or modified the software.
 
\item[3.] You are allowed to add to the header files and the license
  information described in (2b) with changes and your own name, but
  only as an addition at the bottom of the file.
 
\item[4.] Distributing the software must be done in a way so that the
  software archive is intact and no necessary files (except external
  libraries such as MySQL) is always included. The software should be
  compilable after a reasonable set-up of the tool chain and compiler.
 
\item[5.] If you are making modifications to this software i humbly
  suggest you send a copy of your modifications to me on
  ichi@ichimusai.org and I may include your modifications (if you
  allow it) as well as give credit to you (if you want) in the next
  release of the software. This is only a suggestion to keep the code
  base in a single location but it is in no way a restriction to your
  right to modify and redistribute this software.
\end{itemize}


%%%%%%%%%%%%%%%%%%%%%%%%%%%%%%%%%%%%%%%%%%%%%%%%%%%%%%%%%%%%%%%%%%%%%%

\section{Introduction}

I have searched the Internets for a while to find a good logging and
reporting tool for IP tables but most tools were either too complex or
not good enough. I am a firm believer in that security comes from
being clear, simple and transparent and complexity is a very dangerous
path. Therefore I decided to write my own version of a logging tool
that can be used for anyone deploying a Linux server with IP Tables to
get some kind of statistics for what is going on with their machine.
 
Now, this project is still in its initial phase, call it Alpha
software if you like, so there may be lots of bugs and problems,
missing functionality and other things. Some parts of this manual
describes aims and goals rather than actually implemented features. I
hope I will have marked those sections clearly enough.

If you would like to contribute to the software by debugging, writing
code or otherwise maintain, improve upon or develop new features and
so on, please let me know. You can always reach me by email to
$<$anders@sikvall.se$>$.

The project is hosted on GitHub.com for the moment so it is rather
easy to get started and contribute to it.

\subsection{Project aims}

The aims of the project is to create a tool to be able to analyze logs
from the Linux kernel iptables logger and create reports on various
issues, which types of packets are most commonly rejected, suspicious
IP addresses in the world and many other things. It is a rather
ambitious goal and my time is limited but I hope to spend some time on
this and perhaps get a couple of other people involved in this.
 
We should rely as little as possible on external software. Not use
obscure libraries that we are not able to distribute together with the
source or the software and so on. Some things are unavoidable such as
dependence on mySQL in this project wherefore those libraries needs to
be installed but that should be easy to do.
 
Minimal configuration should be required to get up and running. A
quick read through the manual and anyone really that knows their way
around a terminal in Linux should be able to use this.

We do rely on a few outside components, mainly iptables itself and we
will use MySQL as the database to collect the data into for easier
processing. This requires the user to install MySQL if it is not
already installed in the system, and set up usernames and a database
for ipta to be using.

Regarding syslog it is also best to set it up so that it will process 
the log for iptables in a separate log file, but we will explain how 
to configure the most common syslogd in this manual also. Most syslog 
daemons have similar configuration files.

\subsection{Bug reports}

This is just an early pre-release really and there is a lot of things
that are not included yet. You may find a few that are just stubs or
that are not working properly.

Please report any other bugs you find on the issue tracker
at\\ https://github.com/sikvall/ipta it is much appreciated if you
would take the time and describe how to re-create the bug.
 
%%%%%%%%%%%%%%%%%%%%%%%%%%%%%%%%%%%%%%%%%%%%%%%%%%%%%%%%%%%%%%%%%%%%%%

\section{Installation}

This chapter describes the installation of the source package, the
tools necessary to build it, compiling and installation of the
software on the target machine.

\subsection{Installing mysql server and client}

The system relies on mysql server and client software installed on the
local machine or another machine. Consult your system documentation on
how to install the necessary tools.

\begin{verbatim}
# apt-get install mysql-server
\end{verbatim}

\subsection{Installing the necessary tools}

The ipta software is delivered as a software package and needs to be
built for the platform you want to run it on. However the result is a
single executable file which can easily be packaged and deployed in
any software distribution you see fit.

The license in chapter \ref{license} grants you the right to deploy
the software in any situation you seem fit, you are also free to
modify the software and I only ask that if you make substantial
improvements or bug fixes that you send them to me for consideration
into the main software stream so others may benefit from this.

We need to install the GNU Compiler suite (GCC) and the C development
libs for mysql as well as the version control system git. This is done
like this in Ubuntu:

\begin{verbatim}
$ sudo apt-get install gcc libmysqlclient-dev git
\end{verbatim}

\subsection{Download the source}

Get the latest software package from the github repository, currently
hosted at GitHub.com by issuing the following command.

\begin{verbatim}
$ git clone https://github.com/sikvall/ipta
\end{verbatim}

This will give you a directory called "ipta" in the current working 
directory which will bet git initialized and ready for you to work 
with or just compile.

\subsection{Building the source and installing}

You can now compile the software on you local machine by doing the
following:

\begin{verbatim}
$ cd src 
$ make all 
$ make install
\end{verbatim}

You should by now if no errors have been shown have a local binary of
"ipta" that you can start testing with. The last command will install
the ipta binary under /usr/bin in your system with owner root and
proper permissions.

The make install command will ask for your sudo password in order to 
install the binary to /usr/bin. If you do not want that you can run 
ipta straight from the src directory by just typing the following:

\begin{verbatim}
$ ./ipta <options> [arguments]
\end{verbatim}

\subsection{Updating to the latest version}

You can always update to the latest version by pulling the changes 
from the github repository at any time. This is done from the command 
line by changing to the ipta directory and issuing the following 
command:

\begin{verbatim}
$ git pull
\end{verbatim}

This will pull the latest changes. When done you can always compile 
and install an updated version of the binary as described earlier. 
You will also receive the updated manual and any other accompanying 
file.

\end{document}
