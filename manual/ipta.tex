% !TeX encoding = UTF-8
\documentclass[english,twoside,openright,a4paper,12pt]{article}
\usepackage{a4}
\usepackage[ansinew]{inputenc}
\usepackage[english]{babel}
\usepackage{graphics}
\usepackage{dcolumn}
\usepackage{fancyhdr}
\usepackage{amsmath}	
\usepackage{graphicx}
\usepackage{color}
\usepackage[hidelinks]{hyperref}
\usepackage{float} 
\restylefloat{table}
\restylefloat{figure}
\usepackage{longtable}
\usepackage{color}
\usepackage{textcomp}
\usepackage{emptypage}
\usepackage[table]{xcolor}
\usepackage{index}
%\makeindex
%\makeglossary
\newcommand{\hilight}[1]{\colorbox{yellow}{#1}}
\nonfrenchspacing
\renewcommand{\baselinestretch}{1.0}
\clubpenalty=9999    % Not higher!
\widowpenalty=9999   % Not higher!
\setcounter{secnumdepth}{4}
\setcounter{tocdepth}{3}
\newcommand{\titlerule}{\rule{\linewidth}{1pt}}
\renewcommand{\headrulewidth}{1.2pt}
\renewcommand{\footrulewidth}{1.2pt}
\newcommand{\ai}[1]{#1\index{#1}}

\author{Anders Sikvall}
\title{
	\Huge ipta\\[2em]
	IP Tables Log Analyzer}

\begin{document}

\begin{titlepage}
  \thispagestyle{empty}
  \begin{center}
    % Logotype goes here
  \end{center}
  \vspace*{\stretch{1}}
  \begin{center}
    \titlerule\\[3mm]
    \Huge \textsc{ipta\\\Large iptables Log Analyzer}\\[5mm]
    \large \emph{Anders Sikvall}\\
    \begin{center}
      \normalsize ichimusai.org\\
      
    \end{center}
    \titlerule\\
  \end{center}
  \scriptsize \today
  \vspace*{\stretch{2}}
  %    \begin{center}
  %         \normalsize\textcopyright Copyright DeltaNode 2015
  %    \end{center}
\end{titlepage}

\null\vfill\thispagestyle{empty}
\noindent

\begin{center}
	Version 0.1\\[5mm] \textcopyright\ Copyright 2015 Anders Sikvall\\
	http://ichimusai.org/projects/ipta\\
	ichi@ichimusai.org
	\newpage
\end{center}

\cleardoublepage

\tableofcontents

% Turn on more technical formatting, this aint some kind of novella
% you know
\setlength{\parindent}{0pt}
\setlength{\parskip}{1em}

\lhead{\nouppercase{\leftmark}}
\rhead{\nouppercase{\rightmark}}

\setlength{\headheight}{15pt}

\pagestyle{fancy}

%%%%%%%%%%%%%%%%%%%%%%%%%%%%%%%%%%%%%%%%%%%%%%%%%%%%%%%%%%%%%%%%%%%%%%

\newpage
\section*{License}
\label{license}

Copyright (c) 2015, Anders "Ichimusai" Sikvall.\\
Latest revision \today
 
Permission is hereby granted, free of charge, to any person obtaining
a copy of this software and associated documentation files (the
"Software"), to deal in the Software without restriction, including
without limitation the rights to use, copy, modify, merge, publish,
distribute, sublicense, and/or sell copies of the Software, and to
permit persons to whom the Software is furnished to do so, subject to
the following conditions:


\begin{itemize}
\item[1.] The above copyright notice and this permission notice shall
  be included in all copies or substantial portions of the Software.
  It is not allowed to modify this license in any way.
  
\item[2a.] The original header must accompany all software files and
  are not removed in redistribution. It is allowed to add to the
  headers to describe modifications of the software and who has made
  these modifications. I claim no right to your modifications, they
  stand on their own merits.
 
\item[2b.] The license shown when the software is invoked with the
  "--license" option is not removed or modified but must remain the
  same. You may add your name to the bottom of the credit part if you
  have contributed or modified the software.
 
\item[3.] You are allowed to add to the header files and the license
  information described in (2b) with changes and your own name, but
  only as an addition at the bottom of the file.
 
\item[4.] Distributing the software must be done in a way so that the
  software archive is intact and no necessary files (except external
  libraries such as MySQL) is always included. The software should be
  compilable after a reasonable set-up of the tool chain and compiler.
 
\item[5.] If you are making modifications to this software i humbly
  suggest you send a copy of your modifications to me on
  ichi@ichimusai.org and I may include your modifications (if you
  allow it) as well as give credit to you (if you want) in the next
  release of the software. This is only a suggestion to keep the code
  base in a single location but it is in no way a restriction to your
  right to modify and redistribute this software.

\end{itemize}


%%%%%%%%%%%%%%%%%%%%%%%%%%%%%%%%%%%%%%%%%%%%%%%%%%%%%%%%%%%%%%%%%%%%%%

\section{Introduction}

I have searched the Internets for a while to find a good logging and
reporting tool for IP tables but most tools were either too complex or
not good enough. I am a firm believer in that security comes from
being clear, simple and transparent and complexity is a very dangerous
path. Therefore I decided to write my own version of a logging tool
that can be used for anyone deploying a Linux server with IP Tables to
get some kind of statistics for what is going on with their machine.

Now, this project is still in its initial phase, call it Alpha
software if you like, so there may be lots of bugs and problems,
missing functionality and other things. Some parts of this manual
describes aims and goals rather than actually implemented features. I
hope I will have marked those sections clearly enough.

If you would like to contribute to the software by debugging, writing
code or otherwise maintain, improve upon or develop new features and
so on, please let me know. You can always reach me by email to
$<$anders@sikvall.se$>$.

The project is hosted on GitHub.com for the moment so it is rather
easy to get started and contribute to it.


% % % % % % % % % % % % % % % % % % % % % % % % % % % % % % %


\subsection{Project aims}

The aims of the project is to create a tool to be able to analyze logs
from the Linux kernel iptables logger and create reports on various
issues, which types of packets are most commonly rejected, suspicious
IP addresses in the world and many other things. It is a rather
ambitious goal and my time is limited but I hope to spend some time on
this and perhaps get a couple of other people involved in this.

We should rely as little as possible on external software. Not use
obscure libraries that we are not able to distribute together with the
source or the software and so on. Some things are unavoidable such as
dependence on mySQL in this project wherefore those libraries needs to
be installed but that should be easy to do.

Minimal configuration should be required to get up and running. A
quick read through the manual and anyone really that knows their way
around a terminal in Linux should be able to use this.

We do rely on a few outside components, mainly iptables itself and we
will use MySQL as the database to collect the data into for easier
processing. This requires the user to install MySQL if it is not
already installed in the system, and set up usernames and a database
for ipta to be using.

Regarding syslog it is also best to set it up so that it will process
the log for iptables in a separate log file, but we will explain how
to configure the most common syslogd in this manual also. Most syslog
daemons have similar configuration files.

\subsection{Bug reports}

This is just an early pre-release really and there is a lot of things
that are not included yet. You may find a few that are just stubs or
that are not working properly.

Please report any other bugs you find on the issue tracker at
Github\footnote{https://github.com/sikvall/ipta}. It is much 
appreciated if you
would take the time and describe how to re-create the bug.

You can of course also email me and I will create an issue for the 
bug and
see what we can do for a resolution. Please email to 
$<$anders@sikvall.se$>$
for reporting bugs and put ``ipta: [description]" in the header.
 
%%%%%%%%%%%%%%%%%%%%%%%%%%%%%%%%%%%%%%%%%%%%%%%%%%%%%%%%%%%%%%%%%%%%

\section{Installation}

This chapter describes the installation of the source package, the
tools necessary to build it, compiling and installation of the
software on the target machine.

\subsection{Installing mysql server and client}

The system relies on mysql server and client software installed on the
local machine or another machine. Consult your system documentation on
how to install the necessary tools.

For Ubuntu you would likely install something like this:

\small
\begin{verbatim}
$ sudo apt-get install mysql-server mysql-client mysql-common
\end{verbatim}
\normalsize

\subsection{Installing the necessary tools}

The ipta software is delivered as a software package and needs to be
built for the platform you want to run it on. However the result is a
single executable file which can easily be packaged and deployed in
any software distribution you see fit.

The license in chapter \ref{license} grants you the right to deploy
the software in any situation you seem fit, you are also free to
modify the software and I only ask that if you make substantial
improvements or bug fixes that you send them to me for consideration
into the main software stream so others may benefit from this.

We need to install the GNU Compiler suite (GCC) and the C development
libs for mysql as well as the version control system git. This is done
like this in Ubuntu:

\begin{verbatim}
$ sudo apt-get install gcc libmysqlclient-dev git make
\end{verbatim}

\subsection{Download the source}

Get the latest software package from the github repository, currently
hosted at GitHub.com by issuing the following command.

\small
\begin{verbatim}
$ git clone https://github.com/sikvall/ipta
\end{verbatim}
\normalsize

This will give you a directory called "ipta" in the current working
directory which will bet git initialized and ready for you to work
with or just compile.

\subsection{Building the source and installing}

You can now compile the software on you local machine by doing the following:

\begin{verbatim}
$ cd src 
$ make all 
$ make install
\end{verbatim}

You should by now if no errors have been shown have a local binary of
"ipta" that you can start testing with. The last command will install
the ipta binary under /usr/bin in your system with owner root and
proper permissions.

The make install command will ask for your sudo password in order to
install the binary to /usr/bin. If you do not want that you can run
ipta straight from the src directory by just typing the following:

\begin{verbatim}
$ ./ipta <options> [arguments]
\end{verbatim}

\subsection{Making the manual}

The manual is delivered as a PDF and as a \LaTeX\ source this means
you can compile it just like you do with the source code for the
binary.

You need a different set of tools to make the manual from scratch but
for most modern Linux systems you should probably have most of these
tools already installed.

\begin{verbatim}
$ sudo apt-get install texlive
\end{verbatim}

The TeX-Live distribution is the one recommended by TeX User Group and
is the default one in Ubuntu and others. But other \LaTeX\ can of
course be used.

Go to the manual directory and type

\begin{verbatim}
$ make
\end{verbatim}

This will start the make process, it will run "pdflatex ipta.tex"
twice and build you a new pdf with the manual. Most users would
however not need to build the manual but can just look in the provided
pre-built PDF file instead.

You may also make postscript versions, or device independetn versions
for printing and such with \LaTeX\ but it is outside the scope of this
manual to describe how this works.

\subsubsection{Updating to the latest version}

You can always update to the latest version by pulling the changes
from the github repository at any time. This is done from the command
line by changing to the ipta directory and issuing the following
command:

\begin{verbatim} 
$ git pull 
\end{verbatim}

This will pull the latest changes. When done you can always compile
and install an updated version of the binary as described earlier. You
will also receive the updated manual and any other accompanying file.

\subsection{Setting up MySQL}

Setting up MySQL for ipta is rather simple. You need to create a user,
a database and grant all privileges on that database for the ipta
user. If you do not configure ipta differently it will attempt to
connect using the following credentials:

Username: ipta\\
Password: ipta\\

This can be overruled on the command line. With no other arguments it
will try to access an existing database called "ipta" and if no
other name is given the default table is named "logs".

\subsubsection{Create MySQL User}

This assumes you already have MySQL installed and have access to the
root account. During most installs you would have set up the root
account when the MySQL database server was installed.

First you need to log in to MySQL:

\begin{verbatim}
$ mysql -u root -p
\end{verbatim}

This will connect a shell to the local MySQL database with user root
asking you to provide a password. Once logged in you would be
presented with the MySQL prompt. It is then time to create a new user,
a database for ipta and set the permissions to use it.

\small
\begin{verbatim}
mysql> CREATE USER 'ipta'@'localhost' IDENTIFIED BY 'ipta';
mysql> CREATE DATABASE ipta;
mysql> GRANT ALL ON ipta.* TO 'ipta'@'localhost' IDENTIFIED BY 'ipta';
mysql> FLUSH PRIVILEGES;
mysql> quit
Bye.
\end{verbatim}
\normalsize

You can now verify that it works by invoking ipta to create the table 
needed for analysis. This is done by the following command:

\begin{verbatim}
$ ipta --create-table
* Table 'logs' created in database which you may now use.
\end{verbatim}

If you get the mssage that the table was successfully created we are
fine. If you get a message it already exists you are probably fine. If
you get an error, check your MySQL statements again and see that you
did create the ipta user and password is correct and so on.

Now we have everything set up in MySQL and can move on the next part,
setting up iptables and syslog.

\subsection{Setting up syslog}

In many systems the iptables messages will be logged to the same
system log as every other kernel message. This means we will have logs
cluttered with all sorts of messages making it hard to see what is
what.

By breaking out the iptables messages to a separate file we will not
clutter the standard system log but can keep it separate and it also
means the processing of the log file will be more efficient by ipta.

The ipta tool always looks for the "IPT: " prefix on the log line
and will ignore any line that do not start with this. In order to set
up syslog properly for our purpose we need to make some changes to its
configuration file.

In this case we are working with \textit{rsyslogd} which is the
standard syslogger in Ubuntu. If you have a different syslog daemon
you may want to consult the manual to do thesame thing. Most have
fairly similar syntax in their configuration files.

\subsubsection{Make syslog use a separate log file for iptables}

In Ubuntu 14.04 you find the configuration in /etc/rsyslog.d where you
can create a new file called 20-iptables.conf and the content of this
file should be:

\begin{verbatim}
# Syslog iptables to separate file
:msg, contains, "IPT: " -/var/log/iptables.log 
# Remove from the other logs
&~
\end{verbatim}

Save this to the disk as the file /etc/rsyslog.d/20-iptables.conf and
then issue a service restart for the syslog daemon by the command:

\begin{verbatim}
$ sudo service rsyslog restart
\end{verbatim}

It should now restart with the new configuration and hopefully log all
iptables messages to a new file.

\subsubsection{Set up log-rotate for daily or weekly rotate}

Since the logfile has a potential of becoming big, I suggest rotating
it on a daily basis and just keeping what you need. It is probably a
good idea to zip the files older than today and yesterday also to save
space on the disk.

To set this up you need to find your logrotate.d directory, it usually
resides in /etc/logrotate.d/ and if you go there you will find the
configuration files for logrotate. One of them is probably already
made for syslog, in Ubuntu 14.04 it will be called rsyslog. Open and
edit this file carefully with an editor of your choice.

\begin{verbatim}
/var/log/iptables.log
{
    rotate 30
    daily 
    missingok 
    notifempty 
    compress 
    delaycompress 
    sharedscripts 
    postrotate 
        reload rsyslog >/dev/null 2>&1 || true 
    endscript
}
\end{verbatim}

This will rotate your iptables.log every day and keep them 30 days
back. It will perform some housekeeping as well as restarting the
rsyslog (you may need to change that to the syslog in your system) in
order to keep sytems nice. It will also compress all files from day
before yesterday and back while keeping the two latest days, today and
yesterday, uncompressed.

Of course if you want to rotate on a weekly basis or something else,
you can change this accordingly. Insert it into the rsyslog or create
a new one called "ipta" or something like that, then save it. Check
it out a couple of days later that it is actually rotating your logs.

\subsection{Logging loops caution}
\hilight{Move this section later}

As with any complex system there is the potential to have unintended
consequences depending on how you set it up. During development I
found out that it is easy to get into a "logging loop" with the
\texttt{--follow} mode engaged. The problem here is that if you log
certain types of packets, such as DNS requests (port 53) or ssh (port
22) and you use the \texttt{--rdns} option or view the traffic over
SSH the ipta may generate it's own traffic that in its turn creates
even more traffic as it attempts to display the already generated
traffic. I will show you two examples of what not to do.

\subsection{Setting up iptables}

Now it is time to setup the iptables to log what we want. You should
be familiar a little bit with iptables already, you do not have to be
an expert but some familiarity with the basic functions is required
here. If you feel you don't have that please take some time to read
through a few of the various good HOWTO documents that are out there
and familiarize yourself with iptables.

\subsubsection{Creating the logchains}

Basically what we will do is create two new chains in iptables that
can be used for targets from other rules. One is to log packets that
we want to DROP and the other is to log packets we accept but still
wants to log.

We can also elaborate on this by creating a log chain for packets that
are invalid and packets from certain ranges of IP addresses we wish to
block and so on, the possibilities are really endless.

Let's start with creating a log chain for packets that are dropped. We
will at the beginning somewhere, before any rules but after the policy
setting in the iptables script insert something like this:

\small
\begin{verbatim}
iptables -N LOGDROP
iptables -A LOGDROP -j LOG --log-prefix "IPT: DROP " --log-level 7
iptables -A LOGDROP -j DROP
\end{verbatim}
\normalsize

Now every time in your firewall that you will drop a packet, instead
of using the usual target DROP you will replace it with LOGDROP
instead. If you still want to drop some packets without logging you
can of course still use the DROP target.

We will also create a chain to handle packets that are accepted but
logged and it can look like this

\small
\begin{verbatim}
iptables -N LOGACCEPT
iptables -A LOGACCEPT -j LOG --log-prefix "IPT: ACCEPT " --log-level 7
iptables -A LOGACCEPT -j ACCEPT
\end{verbatim}
\normalsize

Now we can see that the prefix we use tells ipta something on what is
going on. First the marker "IPT:" is recognized as this is a line
that should be processed, the next word is a signal word that tells
ipta what has happend. This should be ACCEPT, DROP, INVALID or
something similar. ACCEPT and DROP should be used mainly as they have
special meaning in the processor.

The signal word will be put into the database as the action field and
used in statistics so create as many as you like, keep them short and
to the point and prefer to use the ACCEPT and DROP unless you really
want to distinguish between drops for different reasons.

Some action words used in the past that may give you some idea here:

\begin{table}[H]
\begin{tabular}{ll}
	\textbf{Action} & \textbf{Description  }                        \\ \hline
	DROP            & Used generally for dropped packets            \\
	ACCEPT          & Used generally for accepted packets           \\
	INVALID         & Packets specifically dropped as invalid       \\
	CBLK            & Packets that are CIDR blocked (country block)
\end{tabular}
\end{table}

\subsubsection{A full iptables script}

Normally a firewall configuration is not input manually but rather
there is a script that is executed just after boot or sometimes by a
cron job or similar. This script can take many forms but they
generally do something like this:

\begin{enumerate}
\item Clear all previous rules
\item Set default policies
\item Set rules for accepting traffic to specific services
\item Deny all traffic not accepted
\end{enumerate}


Below is an example iptables script, adapted to the ipta tool as well
that you can use and expand upon yourself for your system. This should
give you a general idea on how to set it up.

To fully explain all things with iptables is outside the scope of this
manual but there are many tutorials and howto-docs on the net that
will help you get started. The one below is rather well documented. It
will also assume anything related to the loop-back interface (lo) is
considered safe, if not, you can take that part out.

\small

\scriptsize \begin{verbatim}
#!/bin/bash

# This is the IP Tables script used for ipta (IP Tables Analyzer)
# version x.x.x If your binary for iptables is located in a different
# directory change this definition

IPTABLES="/sbin/iptables"

# Setting default policies if you want to log dropped packets the
# default policy should be ACCEPT so that you can send them to the log
# facility, otherwise they will be dropped silently

$IPTABLES -P INPUT   ACCEPT
$IPTABLES -P OUTPUT  ACCEPT
$IPTABLES -P FORWARD DENY #Change to ACCEPT ifyou use forwarding

# Flushing existing rules to make sure our iptables are empty 
$IPTABLES -F

# Create the logging chains for our iptables

# What happens here is that we add two rules to each chain. One that
# sends the packet to be logged via syslog using a recognizable prefix
# as well as a single word telling what happened to the packet (DROP
# or ACCEPT) and then the packet is either dropped or accepted.

$IPTABLES -N LOGDROP
$IPTABLES -A LOGDROP -j LOG --log-prefic "IPT: DROP " --log-level 7
$IPTABLES -A LOGDROP -j DROP

$IPTABLES -N LOGACCEPT
$IPTABLES -A LOGACCEPT -j LOG --log-prefix "IPT: ACCEPT " --log-level 7 
$IPTABLES -A LOGACCEPT -j ACCEPT

$IPTABLES -N LOGINVALID
$IPTABLES -A LOGINVALID -j LOG --log-prefix "IPT: INVALID " --log-level 7 
$IPTABLES -A LOGINVALID -j DROP

# Drop early all malformed packets
$IPTABLES -A INPUT -m state --state INVALID -j LOGINVALID

# Allow all traffic on local interfalce "lo" without logging 
$IPTABLES -A INPUT  -i lo -j ACCEPT
$IPTABLES -A OUTPUT -i lo -j ACCEPT

# Enable ICMP but log them
$IPTABLES -A INPUT -p icmp -j LOGACCEPT $IPTABLES -A OUTPUT -p icmp -j LOGACCEPT

# Enable DNS server connections from local host, no logging here DNS
# is specified for both tcp and udp connections but almost always only
# udp is used so you may disable the tcp parts if you like

$IPTABLES -A INPUT  -p tcp --sport domain -j ACCEPT
$IPTABLES -A INPUT  -p udp --sport domain -j ACCEPT
$IPTABLES -A OUTPUT -p tcp --dport domain -j ACCEPT
$IPTABLES -A OUTPUT -p udp --dport domain -j ACCEPT

# If you are using NTP for time synchronization you should have these
# enables, otherwise comment them out. I like to log them also.

$IPTABLES -A INPUT  -p udp --sport 123 -m state --state ESTABLISHED -j LOGACCEPT 
$IPTABLES -A OUTPUT -p udp --sport 123 -m state --state NEW,ESTABLISHED -j LOGACCEPT

# This enables incoming SSH connections to the local host we will log
# this activity. Logging the ESTABLISHED state usually renders a lot
# of packets in the log since that would be file transfers and so on,
# therefore we just log the new packets.

$IPTABLES -A INPUT  -p tcp --dport 22 -m state --state NEW -j LOGACCEPT
$IPTABLES -A INPUT  -p tcp --dport 22 -m state --state ESTABLISHED -j ACCEPT 
$IPTABLES -A OUTPUT -p tcp --sport 22 -m state --state ESTABLISHED -J ACCEPT

# Enable traffic to your web server on port 80, no logging

$IPTABLES -A INPUT  -p tcp --dport 80 -m state --state NEW,ESTABLISHED -j ACCEPT 
$IPTABLES -A OUTPUT -p tcp --sport 80 -m state --state ESTABLISHED -j ACCEPT

# Everything else that falls through here will be considered
# non-wanted traffic and will therefore be logged also, then dropped.

$IPTABLES -A INPUT  -j LOGDROP
$IPTABLES -A OUTPUT -j LOGDROP
\end{verbatim} \normalsize

The above should give you a good starting point that you can modify
into your own needed rules for the firewall you are deploying. You may
of course simplify it or make it as complex as needed but this should
show the geist of how you can use the different log chains to log
packets that you want to analyze later.

Now, this script above creates an iptables firewall, make sure you
have a way to connect to your computer before you run it in case you
loose the connection with it. Don't update firewalls remotely unless
you are certain of what you are doing.

This version makes use of the following actions:

\begin{itemize}
	\item INVALID - packets that are invalid
	\item DROP - when a packet is dropped
	\item ACCEPT - when a packet is accepted in or out of the system
\end{itemize}

You can add others if you like so that you can distinguish between
packets from known hosts etc, the limits are endless. We do recommend
you have these three log chains however since the statistict module in
ipta looks for these three actions.

\subsubsection{Interesting tips \& tricks}

\hilight{Coming later}


\section{Usage and Syntax}

This tool is used to parse log files from syslogd regarding IP-tables
data. Depending on how you set your iptables up to log you may log
only denied, accepted or both types of packets. This is entirely up to
you and you must configure the iptables on your machine to do some
logging first in order to get some data for the ipta import module to
work with.
 
You need to have your MySQL setup for this tool to work. Also,
remember that you may want to clear your existing tables when starting
a new import, otherwise it will just be appended to the earlier data
that you already have in your logs. If you import the same file over
you would skew the results as you would have twice the entrys for the
same packets being logged.
 
You can change the settings for the database in the ipta-system
configuration file in your home folder. It should be called .ipta-conf
and will be read upon start to find your database names, passwords
etc. \hilight{Config file is not yet implemented.}

\subsection{Options}

Any settings in the configuration file can be overridden with command
line parameters. They way ipta decides on the configuration is that
first it will use the ones given on the command line. If no options
are given it will check the configuration file and if there is no
config file it will use default settings.


\scriptsize
\begin{longtable}{|p{.3\textwidth}|p{.65\textwidth}|}
\hline
\textbf{Mode switch}& \textbf{Description}\\\hline

\texttt{-c, --clear} & 

Clears all entries in the database. Can be used in front of the import
directive to clear the database before importing new data.\\\hline

\texttt{-i, --import $<$file$>$} & 

Import a file into the database. If the database already contains data
the new data will be appended to the existing. If you do not wish to
add to the data use the -c or --clear directive in front of the import
directive in order to clear first, then import.\\\hline

\texttt{-a, --analyze} &  

This is a mode switch and tells ipta to do the automatic analysis
that's built in to the software. The output is sent to standard
output. \\\hline

\texttt{-f, --folow $<$file$>$} & 

This changes the behaviour of ipta into a real-time analyzer. It will
follow the log file and any new packets that are logged will be
presented as a line on the screen to stdandard output. Every 20 lines
a new header is written detailing the fields to aid the operator
idetinfyin what he sees on the screen.

This is useful for real-time monitoring of any logged packets. It will
not affect the database in any way or store any data so all the switches and 
parameters dealing with that has no effect when this mode
is employed.\\\hline

\texttt{-ai, --analyze-interactive} & \hilight{Not yet implemented.}
In the future this will allow you to open a shell and put custom
queries to the database in such a way that you can create your own
analysis.\\\hline

\end{longtable}

\begin{longtable}{|p{.3\textwidth}|p{.65\textwidth}|}
\hline
\textbf{Database options} & \textbf{Description}\\ \hline

\texttt{-d, --db-name $<$name$>$} & Use a different database called
name instead of the one in the configuration file or defaul one
(ipta). Useful when you are running multiple analysis in the same
MySQL database or for some other reason can not use the default
name.\\\hline

\texttt{-h, --db-host $<$host$>$} & 

Use a different host name for the
MySQL database. If not, ipta assumes localhost as the host for the
MySQL database.\\\hline

\texttt{-u, --db-user $<$name$>$} & 

Username that ipta uses to connecto to the MySQL database.. This
overrides the name in specification file and the default name "ipta".
\\\hline

\texttt{-p, --db-pass $<$pass$>$} & 

Password to use for connecting to the MySQL database. If this is not
given, the configuration file password will be used or if that is not
designed the default password "ipta" will be tried.\\\hline

\texttt{-pi, --db-pass-i} & 

Asks for the pasword interactively, otherwise the same thing as
\texttt{-p, --db-pass} will do. The reason to use this one rather than
the other one is that passwords entered on a command line may show up
in the process table and rea not considered secure at all.\\\hline

\texttt{-t, --db-table $<$table$>$} & 

Use the table named "table" instead of the default one or the one in
the configuration file. The default table name that ipta works with
would be "logs".\\\hline

\texttt{-lt, --list-tables} & 

List all the tables in the database ipta is using. This is useful if
you work through different tables and want to list their
names.\\\hline

\texttt{-dt, --delete-table $<$table$>$} & 

Remove any unwanted "table" from the database. The table and all
information is erased. This can not be undone.\\\hline

\texttt{-da, --delelte-all-tables} & 

Remove all tables from the database. This will unceremoniously delete
all the data in the ipta database.\\\hline

\texttt{-ct, --create-table [name]} & 

Create a new ipta table. This could be the default table or you can
supply an argument to give the new table a different name. \\\hline

\texttt{-s, --save-db} & 

\hilight{Not yet implemented.} This option will write out a
configuration file to the user's home directory named .ipta with the
current settings. By providing configuration of database name, user,
password and table to use as default these setting will be written to
a file and used as default settings next time you run ipta.\\\hline
\end{longtable}
\normalsize

\scriptsize
\begin{longtable}{|p{.3\textwidth}|p{.65\textwidth}|}
\hline
\textbf{Format switches} & \textbf{Description}\\ \hline

\texttt{-r, --rdns} & 

Do a reverse DNS lookup on the IP addresses that are being analyzed.
This will in the reports replace the IP numbers usually seen with the
DNS names for those addresses (if they exist). If no reverse DNS
record exist, the IP address will be shown instead. The --rdns flag
will sometimes give you good information about the host sending the
packets.\\\hline

\texttt{-l, --limit $<$num$>$} & 

The standard number of lines presented in the analyzer module of ipta
is 10 lines per analyzer submodule. With this parameter you can change
that to $<$num$>$ lines instead of the default.\\\hline

\texttt{--license} & 

Print the license information to standard output. The license is a
modified version of the MIT license giving you the right to do
anything with the software as long as you keep the
attribution.\\ \hline

\texttt{--no-header} & 

This switch removes the headers from the output when the
\texttt{--follow} mode is being employed.\\\hline

\texttt{--no-counter} & 

Removes the line counter in front of the lines in \texttt{--follow}
mode.\\\hline

\texttt{--no-local} & 

This switch removes the local inteface "lo" from the statistics when
interpreting the data in the analyzer mode.\\\hline

\texttt{--no-accept} & 

This switch removes all packets marked with the action "ACCEPT" from
the analyzer output and focuses the analysis on the dropped and
invalid packets.\\\hline
\end{longtable}

\scriptsize
\begin{longtable}{|p{.3\textwidth}|p{.65\textwidth}|}
\hline
\textbf{Miscellaneous} & \textbf{Description}\\ \hline

\texttt{--license} &
Print the license and then exit.\\\hline

\texttt{--version} &
Print version information then exit.\\\hline

\texttt{--usage} &
Print quick usage instructions then exit.\\\hline

\end{longtable}
\normalsize

\section{How to use ipta}

The first thing you need to do is set up your database, iptables log
and logrotate as previously described in this manual. Once you have
that working it is time to use the tool.

We can start by initializing the database:

\begin{verbatim}
$ ipta --create-table
\end{verbatim}

This will create the default table in the database and prime it ready
for use. The next step is to import the current iptables log file into
the MySQL database where it can be later easily analyzed.

This is done by the import mode and triggered by the mode switch
\texttt{--import} followed by the file name. It could be something like this:

\begin{verbatim}
$ ipta --import /var/log/iptables.log
\end{verbatim}

Once the log has been properly imported, and the progress will be
shown during the import which will sometimes take several minutes if
it is a big iptables logfile.

Now we can analyze the imported data by taking a look at it with
ipta's buildt in analyzer module. This is triggered by the mode switch
\texttt{--analyze} and will show you various outputs depending on the
packets you have logged.

\section{Analyzer modules output}

The analyzer runs a number of automated queries on the MySQL database
and then collects the results and formats it in a visual pleasing way
so it is easy to contrast and compare and keep track of what goes on
in the system. This part will describe the different modules used
today.

\subsection{Denied traffic grouped by IP, destport and action}

This module shows everything that is NOT marked as ``ACCEPT" and
therefore is assumed denied. It then groups it by IP address,
destination port number and which action was taken with the packet. It
is useful for an overlook of what kind of traffic we get that gets
denied.

By default the top ten lines are shown, this can be changed with the
\texttt{--limit} option.

\scriptsize \begin{verbatim}
Showing denied traffic grouped by IP, destination port, action taken and protocol.
 Count Source IP                 SPort Dest IP                   DPort Proto  Action
------ ------------------------- ----- ------------------------- ----- ------ ----------
   138 213.136.38.8              41757 188.126.93.160               80 TCP    INVALID   
    14 125.16.128.122            62215 188.126.93.160               80 TCP    INVALID   
    14 204.124.183.98            53512 188.126.93.160               80 TCP    INVALID   
    14 91.200.12.11              56310 188.126.93.160               80 TCP    INVALID   
    10 194.71.19.244             48725 188.126.93.160               80 TCP    INVALID   
     9 176.111.61.12             50139 188.126.93.160               80 TCP    INVALID   
     9 117.21.176.95              6000 188.126.93.160             1433 TCP    DROP      
     9 71.176.122.34              8235 188.126.93.160               80 TCP    INVALID   
     9 176.61.140.118             4080 188.126.93.160               25 TCP    DROP      
     8 194.63.142.101            36697 188.126.93.160              110 TCP    DROP      
     8 201.157.0.78              50608 188.126.93.160               80 TCP    INVALID   
     7 121.155.129.180           48461 188.126.93.160            10142 UDP    DROP      
     6 171.96.241.142            45479 188.126.93.160               25 TCP    DROP      
     6 194.153.119.59            63395 188.126.93.160               80 TCP    INVALID   
     5 200.5.112.174              3163 188.126.93.160               80 TCP    INVALID   
     4 222.186.31.137             6000 188.126.93.160             1433 TCP    DROP      
     4 176.37.170.58             40466 188.126.93.160             8080 TCP    DROP      
     4 162.244.35.24             41695 188.126.93.160            21320 TCP    DROP      
     3 212.162.17.234             3371 188.126.93.160               23 TCP    DROP      
     3 222.186.190.157           43484 188.126.93.160            20022 TCP    DROP  
\end{verbatim} \normalsize

\subsection{ICMP module}

The next module shows the ICMP traffic grouped by IP addres and
action. This is usefull since it is both an indication if there is a
problem to a certain host as well as if there is somebody spraying the
host with ICMP packets such as various ping scans and so on.

This module shows all traffic logged with protocol ICMP no matter the
status.

\scriptsize \begin{verbatim}
Showing ICMP traffic statistics
 Count Source IP                 Dest IP                   Action    
------ ------------------------- ------------------------- ----------
    14 188.126.93.160            129.82.138.44             ACCEPT    
     3 128.9.168.98              188.126.93.160            ACCEPT    
     3 203.178.148.19            188.126.93.160            ACCEPT    
     3 129.82.138.44             188.126.93.160            ACCEPT    
     2 195.251.255.69            188.126.93.160            ACCEPT    
     1 190.7.215.194             188.126.93.160            ACCEPT    
     1 14.8.249.133              188.126.93.160            ACCEPT    
     1 124.65.193.150            188.126.93.160            ACCEPT    
\end{verbatim} \normalsize

\subsection{Most denied ports module}

This part shows grouped on port number and action what goes on and is
sorted top-down showing the most denied ports first. This is useful to
determine if any certain services are being attacked or what is going
on in connection attempts.

\scriptsize \begin{verbatim}
Most denied ports
 Count   DPort   Proto    Action       
------   -----   ------   ----------   
   234       80   TCP      INVALID   
    56       23   TCP      DROP      
    22       25   TCP      DROP      
    14     1433   TCP      DROP      
    13       21   TCP      DROP      
    10     8080   TCP      DROP      
     9      110   TCP      DROP      
     9     3389   TCP      DROP      
     7     5060   UDP      DROP      
     7    10142   UDP      DROP      
\end{verbatim} \normalsize

\subsection{Invalid packets source}

This module sorts invalid packets (action = ``INVALID") based on the
number of packets, their origin and also shows source and destination
ports for the packets.

It is useful to determine network problems as well as malformed
packets or source routed spoofed packets.

\scriptsize \begin{verbatim}Most invalid packets comes from
 Count   Source IP                   SPort   Dest IP                     DPort   Proto    
------   -------------------------   -----   -------------------------   -----   ------   
   138   213.136.38.8                41757   188.126.93.160                 80   TCP      
    14   125.16.128.122              62215   188.126.93.160                 80   TCP      
    14   91.200.12.11                56310   188.126.93.160                 80   TCP      
    14   204.124.183.98              53512   188.126.93.160                 80   TCP      
    10   194.71.19.244               48725   188.126.93.160                 80   TCP      
     9   176.111.61.12               50139   188.126.93.160                 80   TCP      
     9   71.176.122.34                8235   188.126.93.160                 80   TCP      
     8   201.157.0.78                50608   188.126.93.160                 80   TCP      
     6   194.153.119.59              63395   188.126.93.160                 80   TCP      
     5   200.5.112.174                3163   188.126.93.160                 80   TCP      
     3   8.2.122.80                  28016   188.126.93.160              38267   TCP      
     3   109.190.104.220             52561   188.126.93.160                 80   TCP      
     3   177.91.97.84                50385   188.126.93.160                 80   TCP      
     2   104.28.11.14                   80   188.126.93.160              65399   TCP      
     1   178.63.22.198               19505   188.126.93.160              54699   TCP      
     1   195.198.171.84              49473   188.126.93.160                 80   TCP      
     1   108.167.143.218                80   188.126.93.160              28498   TCP      
\end{verbatim} \normalsize

\subsection{Interface statistics}

This module shows the packets count per action and interface. It is
useful as a marker and keep an eye on things in Interface statistics
general.

\footnotesize \begin{verbatim}
 Count   IF In        Action       Proto
------   ----------   ----------   -----
   241   eth0         INVALID      TCP      
   217   eth0         DROP         TCP      
    43   eth0         DROP         UDP      
\end{verbatim} \normalsize

\section{Follow mode}

The ipta follow mode is special because it shows real-time, or at
least close to real-time what goes on when the system is running.
Every packet logged will be subject to ipta's formatting functions and
displayed on a single line. 

You need a terminal with wide lines to make proper use of this but
that should generally not be too much of a problem in this day and
age.

The follow mode is iniated by the following command:

\begin{verbatim}
$ ipta --follow /var/log/iptables.log
\end{verbatim}

You may also want to combine this with the switches \texttt{--no-lo}
to remove the packets to and from the local interface as this is
traffic that is internal on the host and perhaps also combine it with
the \texttt{--no-accept} switch if you only wish to see denied
traffic.

Another very useful switch would of course be the \texttt{--rdns}
switch which will reverse look-up the host name if it exists of the
source and destination IP of the address.

The follow mode shows most of the interesting parameters in the table
such as source IP and port as well as destination IP and port and the
action taken on the packet. With the \texttt{--no-accept} flag the
system will skip any line with the action set to ACCEPT of course.

\tiny \begin{verbatim}
Count    IF       Source                          Port Destination                     Port Proto      Action    
-------- -------- ------------------------------ ----- ------------------------------ ----- ---------- ----------
    1951 eth0     *6f44.cust.bredbandsbolaget.se 35396 zathras.ichimusai.org          15003 TCP        DROP      
    1952 eth0     *6f44.cust.bredbandsbolaget.se 50758 zathras.ichimusai.org           6001 TCP        DROP      
    1953 eth0     *6f44.cust.bredbandsbolaget.se 45937 zathras.ichimusai.org           1580 TCP        DROP      
    1954 eth0     *6f44.cust.bredbandsbolaget.se 48118 zathras.ichimusai.org          42510 TCP        DROP      
    1955 eth0     *6f44.cust.bredbandsbolaget.se 47564 zathras.ichimusai.org           3871 TCP        DROP      
    1956 eth0     *6f44.cust.bredbandsbolaget.se 36463 zathras.ichimusai.org            563 TCP        DROP      
    1957 eth0     *6f44.cust.bredbandsbolaget.se 41547 zathras.ichimusai.org           1066 TCP        DROP      
    1958 eth0     *6f44.cust.bredbandsbolaget.se 41956 zathras.ichimusai.org           1287 TCP        DROP      
    1959 eth0     *6f44.cust.bredbandsbolaget.se 50023 zathras.ichimusai.org           9917 TCP        DROP      
    1960 eth0     *6f44.cust.bredbandsbolaget.se 59197 zathras.ichimusai.org          50001 TCP        DROP      
    1961 eth0     *6f44.cust.bredbandsbolaget.se 53897 zathras.ichimusai.org           1272 TCP        DROP      
    1962 eth0     *6f44.cust.bredbandsbolaget.se 34387 zathras.ichimusai.org           1112 TCP        DROP      
    1963 eth0     *6f44.cust.bredbandsbolaget.se 52943 zathras.ichimusai.org           1218 TCP        DROP      
    1964 eth0     *6f44.cust.bredbandsbolaget.se 37205 zathras.ichimusai.org           1185 TCP        DROP      
    1965 eth0     *6f44.cust.bredbandsbolaget.se 42648 zathras.ichimusai.org           1175 TCP        DROP      
    1966 eth0     *6f44.cust.bredbandsbolaget.se 40920 zathras.ichimusai.org            683 TCP        DROP      
    1967 eth0     *6f44.cust.bredbandsbolaget.se 36787 zathras.ichimusai.org          15660 TCP        DROP      
    1968 eth0     *6f44.cust.bredbandsbolaget.se 41813 zathras.ichimusai.org           2525 TCP        DROP      
    1969 eth0     *6f44.cust.bredbandsbolaget.se 57365 zathras.ichimusai.org           1108 TCP        DROP      
    1970 eth0     *6f44.cust.bredbandsbolaget.se 48566 zathras.ichimusai.org           8443 TCP        DROP      
\end{verbatim} \normalsize
\end{document} 
